%%%%%%%%%%%%%%%%%%%%%%%%%%%%%%%% Configuration du document %%%%%%%%%%%%%%%%%%%%%%%%%%%%%%%
%%% Mise en page et format
\documentclass[a4paper, 10pt]{article} % Article au format A4
\usepackage[top=2cm, bottom=2cm,
			left=2cm, right=2cm,
			headheight=18pt]{geometry} % Marges

%%% Infos
\title{Projet Base de Données Compte rendu}
\author{Adam BENOIT - Anaïs DOUET - Mathieu MEUNIER - Pierre POINAS - Steve LENING}
\date{01 décembre 2023}

%%% Utilisation des infos title, author et date
\makeatletter
\let\Titre\@title
\let\Auteurs\@author
\let\Date\@date
\let\Ecole\@institute
\makeatother

%%% Police de caractères et langue
\usepackage[french]{babel} %Document en français
\usepackage{lmodern}
\usepackage[utf8]{inputenc}
\usepackage[T1]{fontenc}
\setlength{\parindent}{0pt}
\usepackage{xspace} %Espaces corrects avec babel

%%% Configuration des en-têtes et pieds de page
\usepackage{fancyhdr} % En-têtes et pieds de page
\pagestyle{fancy}
\fancyhead[L]{Compte rendu}
\fancyhead[C]{Projet Base de Données}
\fancyhead[R]{ENSIMAG 2A ISI}

%%% Configuration de la table des matières
\renewcommand{\thesection}{\Roman{section}.}
\renewcommand{\thesubsection}{\arabic{subsection}.}
\renewcommand{\thesubsubsection}{\alph{subsubsection})}

%%%Gestion des images
\usepackage{graphicx} %Intégration d'images
\graphicspath{{Figs/}}

%%% Gestions des couleurs
\usepackage[svgnames]{xcolor}

%%% Mise en page
\usepackage{pdflscape} %Mettre certaines pages au format paysage

%%% Gestion des listes
\usepackage{enumitem}

\usepackage{tikz}
\usepackage{svg}
\usepackage{amsmath}
\usepackage{amssymb}

\newcommand{\idee}[1]{\textcolor{blue}{#1}}








%%%%%%%%%%%%%%%%%%%%%%%%%%%%%%%%%%% Contenu du document %%%%%%%%%%%%%%%%%%%%%%%%%%%%%%%%%%
\begin{document}
%%%%%%%%%%%%%%%%%% Page de garde %%%%%%%%%%%%%%%%%
\begin{titlepage}
\begin{center}
  %%% Titre
  \color{Navy}
  \fontsize{35pt}{12.8pt}
  \textbf{\Titre}\\
  \vspace{20pt}
  %%% Sous-titre
  \color{Black}
  \Huge
  \textbf{Club de montagne}
  \vfill
  %%% Illustration
  \includegraphics[width=0.9\textwidth]{logo_database-oracle.jpg} 
  \vfill
  %%% Auteurs
  \normalsize
  \textbf{\Auteurs}\\
  Grenoble INP - Ensimag, UGA \\
  Année 2023-2024
\end{center}
\end{titlepage}



%%%%%%%%%%%%%%% Table des matières %%%%%%%%%%%%%%%
\section*{Contexte du projet}
Un club d'activités de montagne souhaite mettre en place une base de données afin de permettre la gestion des services proposés à leurs adhérents et au grand public. Le service proposé est simple :
\begin{itemize}[label=$\bullet$]
    \item l'ensemble des utilisateurs inscrits sur ce système ont la possibilité de faire des réservations de repas/nuits dans les refuges gérés par le club ;
    \item seuls les adhérents du club ont la possibilité de s'inscrire à des formations et de réserver du matériel pour leurs sorties en montagne. 
\end{itemize}
Les caractéristiques du projet à implémenter sont décrites de manière plus exhaustives dans le sujet.



\vspace{2em}

\tableofcontents

\newpage





%%%%%%%%%%%%%% Remarques techniques %%%%%%%%%%%%%%
\section{Analyse}

\begin{tabular}{rlll}
  mail\_refuge (str)  & $\rightarrow$ & nom\_refuge & (str) \\
               & $\nrightarrow$ & telephone\_refuge & (str) \\
               & $\rightarrow$ & secteur\_geographique & (str) \\
              & $\rightarrow$ & presentation & (str) \\
               & $\rightarrow$ & date\_ouverture & (date) \\
               & $\rightarrow$ & date\_fermeture & (date) \\
               & $\rightarrow$ & places\_repas & (int) \\
              & $\rightarrow$ & places\_nuit & (int) \\
              & $\rightarrow$ & prix\_nuitee & (int) \\
\\
  mail\_refuge & $\twoheadrightarrow$ & type\_moyen\_paiement & (str) \\
  mail\_refuge & $\twoheadrightarrow$ & type\_repas & (str) \\
  \\
  mail\_refuge, type\_repas (str) & $\rightarrow$ & prix & (int) \\
  \\
  id\_formation (str) & $\rightarrow$ & nom\_formation & (str) \\
               & $\rightarrow$ & date\_formation & (date) \\
               & $\rightarrow$ & duree\_formation & (int) \\
               & $\rightarrow$ & description\_formation & (str) \\
               & $\rightarrow$ & places\_formation & (int) \\
               & $\rightarrow$ & prix\_formation & (int) \\
  \\
  id\_formation & $\twoheadrightarrow$ & type\_activites & (str) \\
  \\
  marque (str), modèle (str), date\_achat (date) & $\rightarrow$ & categorie & (str) \\
               & $\rightarrow$ & nb\_pieces & (int) \\
               & $\rightarrow$ & prix\_casse & (int) \\
               & $\rightarrow$ & description\_materiel & (str) \\
               & $\rightarrow$ & date\_peremption & (date) \\
  \\
  marque, modèle, date\_achat & $\twoheadrightarrow$ & type\_activites & (str) \\
  \\
  categorie (str) & $\rightarrow$ & categorie\_parente & (str) \\
  \\
  id\_utilisateur (int) & $\rightarrow$ & somme\_due & (int) \\
                        & $\rightarrow$ & somme\_remboursee & (int) \\
                        & $\nrightarrow$ & id\_adherent & (int) \\
  \\
  mail\_membre (str) & $\rightarrow$ & mail\_membre & (str) \\ 
                        & $\rightarrow$ & nom\_membre & (str) \\
                        & $\rightarrow$ & mot\_de\_passe & (str) \\
\\
id\_reservation\_refuge (int) & $\rightarrow$ & date\_reservation\_ref & (date) \\
                        & $\rightarrow$ & heure\_reservation\_ref & (int) \\
                        & $\rightarrow$ & prix\_total\_reservation & (int) \\
\\
id\_reservation\_refuge, jour (int) & $\not\twoheadrightarrow$ & type\_repas & (str) \\
\\
id\_reservation\_formation (int) & $\rightarrow$ & rang\_attente & (int) \\
\\
id\_reservation\_materiel (int) & $\rightarrow$ & nb\_pieces & (int) \\
                                & $\rightarrow$ & date\_recup & (date) \\
                                & $\rightarrow$ & date\_rendu & (date) \\
                                & $\nrightarrow$ & nb\_casse & (int) \\

\end{tabular}

\begin{landscape} \pagestyle{empty}
\section{Conception}
    \begin{figure}[h!]
      %\vspace{-2em}
      \includegraphics[width=\linewidth]{diagram.png}
      \caption{Représentation du premier étage (seul) dans le carter}
    \end{figure}
\end{landscape}



\section{Traduction en relationnel} % Implémentation dans Oracle

\section{Requêtes \& Transactions}

\section{Mode d'emploi du démonstrateur}

\section{Bilan}
\subsection{Organisation},
\begin{enumerate}
    \item \textbf{Analyse :} Tout le groupe
    \begin{itemize}
        \item Analyse statique
        \item Dépendances fonctionnelles
        \item Détermination des contraintes (valeur, multiplicité, contextuelles)
    \end{itemize}
    \item 
\end{enumerate}
\subsection{Point difficiles rencontrés}




\end{document}